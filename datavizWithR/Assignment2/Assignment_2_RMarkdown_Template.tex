% Options for packages loaded elsewhere
\PassOptionsToPackage{unicode}{hyperref}
\PassOptionsToPackage{hyphens}{url}
\PassOptionsToPackage{dvipsnames,svgnames,x11names}{xcolor}
%
\documentclass[
]{article}
\usepackage{amsmath,amssymb}
\usepackage{iftex}
\ifPDFTeX
  \usepackage[T1]{fontenc}
  \usepackage[utf8]{inputenc}
  \usepackage{textcomp} % provide euro and other symbols
\else % if luatex or xetex
  \usepackage{unicode-math} % this also loads fontspec
  \defaultfontfeatures{Scale=MatchLowercase}
  \defaultfontfeatures[\rmfamily]{Ligatures=TeX,Scale=1}
\fi
\usepackage{lmodern}
\ifPDFTeX\else
  % xetex/luatex font selection
\fi
% Use upquote if available, for straight quotes in verbatim environments
\IfFileExists{upquote.sty}{\usepackage{upquote}}{}
\IfFileExists{microtype.sty}{% use microtype if available
  \usepackage[]{microtype}
  \UseMicrotypeSet[protrusion]{basicmath} % disable protrusion for tt fonts
}{}
\makeatletter
\@ifundefined{KOMAClassName}{% if non-KOMA class
  \IfFileExists{parskip.sty}{%
    \usepackage{parskip}
  }{% else
    \setlength{\parindent}{0pt}
    \setlength{\parskip}{6pt plus 2pt minus 1pt}}
}{% if KOMA class
  \KOMAoptions{parskip=half}}
\makeatother
\usepackage{xcolor}
\usepackage[margin=1in]{geometry}
\usepackage{color}
\usepackage{fancyvrb}
\newcommand{\VerbBar}{|}
\newcommand{\VERB}{\Verb[commandchars=\\\{\}]}
\DefineVerbatimEnvironment{Highlighting}{Verbatim}{commandchars=\\\{\}}
% Add ',fontsize=\small' for more characters per line
\usepackage{framed}
\definecolor{shadecolor}{RGB}{248,248,248}
\newenvironment{Shaded}{\begin{snugshade}}{\end{snugshade}}
\newcommand{\AlertTok}[1]{\textcolor[rgb]{0.94,0.16,0.16}{#1}}
\newcommand{\AnnotationTok}[1]{\textcolor[rgb]{0.56,0.35,0.01}{\textbf{\textit{#1}}}}
\newcommand{\AttributeTok}[1]{\textcolor[rgb]{0.13,0.29,0.53}{#1}}
\newcommand{\BaseNTok}[1]{\textcolor[rgb]{0.00,0.00,0.81}{#1}}
\newcommand{\BuiltInTok}[1]{#1}
\newcommand{\CharTok}[1]{\textcolor[rgb]{0.31,0.60,0.02}{#1}}
\newcommand{\CommentTok}[1]{\textcolor[rgb]{0.56,0.35,0.01}{\textit{#1}}}
\newcommand{\CommentVarTok}[1]{\textcolor[rgb]{0.56,0.35,0.01}{\textbf{\textit{#1}}}}
\newcommand{\ConstantTok}[1]{\textcolor[rgb]{0.56,0.35,0.01}{#1}}
\newcommand{\ControlFlowTok}[1]{\textcolor[rgb]{0.13,0.29,0.53}{\textbf{#1}}}
\newcommand{\DataTypeTok}[1]{\textcolor[rgb]{0.13,0.29,0.53}{#1}}
\newcommand{\DecValTok}[1]{\textcolor[rgb]{0.00,0.00,0.81}{#1}}
\newcommand{\DocumentationTok}[1]{\textcolor[rgb]{0.56,0.35,0.01}{\textbf{\textit{#1}}}}
\newcommand{\ErrorTok}[1]{\textcolor[rgb]{0.64,0.00,0.00}{\textbf{#1}}}
\newcommand{\ExtensionTok}[1]{#1}
\newcommand{\FloatTok}[1]{\textcolor[rgb]{0.00,0.00,0.81}{#1}}
\newcommand{\FunctionTok}[1]{\textcolor[rgb]{0.13,0.29,0.53}{\textbf{#1}}}
\newcommand{\ImportTok}[1]{#1}
\newcommand{\InformationTok}[1]{\textcolor[rgb]{0.56,0.35,0.01}{\textbf{\textit{#1}}}}
\newcommand{\KeywordTok}[1]{\textcolor[rgb]{0.13,0.29,0.53}{\textbf{#1}}}
\newcommand{\NormalTok}[1]{#1}
\newcommand{\OperatorTok}[1]{\textcolor[rgb]{0.81,0.36,0.00}{\textbf{#1}}}
\newcommand{\OtherTok}[1]{\textcolor[rgb]{0.56,0.35,0.01}{#1}}
\newcommand{\PreprocessorTok}[1]{\textcolor[rgb]{0.56,0.35,0.01}{\textit{#1}}}
\newcommand{\RegionMarkerTok}[1]{#1}
\newcommand{\SpecialCharTok}[1]{\textcolor[rgb]{0.81,0.36,0.00}{\textbf{#1}}}
\newcommand{\SpecialStringTok}[1]{\textcolor[rgb]{0.31,0.60,0.02}{#1}}
\newcommand{\StringTok}[1]{\textcolor[rgb]{0.31,0.60,0.02}{#1}}
\newcommand{\VariableTok}[1]{\textcolor[rgb]{0.00,0.00,0.00}{#1}}
\newcommand{\VerbatimStringTok}[1]{\textcolor[rgb]{0.31,0.60,0.02}{#1}}
\newcommand{\WarningTok}[1]{\textcolor[rgb]{0.56,0.35,0.01}{\textbf{\textit{#1}}}}
\usepackage{graphicx}
\makeatletter
\def\maxwidth{\ifdim\Gin@nat@width>\linewidth\linewidth\else\Gin@nat@width\fi}
\def\maxheight{\ifdim\Gin@nat@height>\textheight\textheight\else\Gin@nat@height\fi}
\makeatother
% Scale images if necessary, so that they will not overflow the page
% margins by default, and it is still possible to overwrite the defaults
% using explicit options in \includegraphics[width, height, ...]{}
\setkeys{Gin}{width=\maxwidth,height=\maxheight,keepaspectratio}
% Set default figure placement to htbp
\makeatletter
\def\fps@figure{htbp}
\makeatother
\setlength{\emergencystretch}{3em} % prevent overfull lines
\providecommand{\tightlist}{%
  \setlength{\itemsep}{0pt}\setlength{\parskip}{0pt}}
\setcounter{secnumdepth}{-\maxdimen} % remove section numbering
\ifLuaTeX
  \usepackage{selnolig}  % disable illegal ligatures
\fi
\usepackage{bookmark}
\IfFileExists{xurl.sty}{\usepackage{xurl}}{} % add URL line breaks if available
\urlstyle{same}
\hypersetup{
  pdftitle={Assignment 2},
  pdfauthor={Jason Thomas (s3907634)},
  colorlinks=true,
  linkcolor={Maroon},
  filecolor={Maroon},
  citecolor={Blue},
  urlcolor={blue},
  pdfcreator={LaTeX via pandoc}}

\title{Assignment 2}
\usepackage{etoolbox}
\makeatletter
\providecommand{\subtitle}[1]{% add subtitle to \maketitle
  \apptocmd{\@title}{\par {\large #1 \par}}{}{}
}
\makeatother
\subtitle{Deconstruct, Reconstruct Web Report}
\author{Jason Thomas (s3907634)}
\date{}

\begin{document}
\maketitle

\subsubsection{Assessment declaration
checklist}\label{assessment-declaration-checklist}

Please carefully read the statements below and check each box if you
agree with the declaration. If you do not check all boxes, your
assignment will not be marked. If you make a false declaration on any of
these points, you may be investigated for academic misconduct. Students
found to have breached academic integrity may receive official warnings
and/or serious academic penalties. Please read more about academic
integrity
\href{https://www.rmit.edu.au/students/student-essentials/assessment-and-exams/academic-integrity}{here}.
If you are unsure about any of these points or feel your assessment
might breach academic integrity, please contact your course coordinator
for support. It is important that you DO NOT submit any assessment until
you can complete the declaration truthfully.

\textbf{By checking the boxes below, I declare the following:}

\begin{itemize}
\item
  I have not impersonated, or allowed myself to be impersonated by, any
  person for the purposes of this assessment
\item
  This assessment is my original work and no part of it has been copied
  from any other source except where due acknowledgement is made. Due
  acknowledgement means the following:

  \begin{itemize}
  \tightlist
  \item
    The source is correctly referenced in a reference list
  \item
    The work has been paraphrased or directly quoted
  \item
    A citation to the original work's reference has been included where
    the copied work appears in the assessment.
  \end{itemize}
\item
  No part of this assessment has been written for me by any other person
  except where such collaboration has been authorised by the
  lecturer/teacher concerned.
\item
  I have not used generative ``AI'' tools for the purposes of this
  assessment.
\item
  Where this work is being submitted for individual assessment, I
  declare that it is my original work and that no part has been
  contributed by, produced by or in conjunction with another student.
\item
  I give permission for my assessment response to be reproduced,
  communicated, compared and archived for the purposes of detecting
  plagiarism.
\item
  I give permission for a copy of my assessment to be retained by the
  university for review and comparison, including review by external
  examiners.
\end{itemize}

\textbf{I understand that:}

\begin{itemize}
\item
  Plagiarism is the presentation of the work, idea or creation of
  another person or machine as though it is your own. It is a form of
  cheating and is a very serious academic offence that may lead to
  exclusion from the University. Plagiarised material can be drawn from,
  and presented in, written, graphic and visual form, including
  electronic data and oral presentations. Plagiarism occurs when the
  origin of the material used is not appropriately cited.
\item
  Plagiarism includes the act of assisting or allowing another person to
  plagiarise or to copy my work.
\end{itemize}

\textbf{I agree and acknowledge that:}

\begin{itemize}
\item
  I have read and understood the Declaration and Statement of Authorship
  above.
\item
  If I do not agree to the Declaration and Statement of Authorship in
  this context and all boxes are not checked, the assessment outcome is
  not valid for assessment purposes and will not be included in my final
  result for this course.
\end{itemize}

\textbf{This is a template file. The following example included is not
considered a good example to follow for Assignment 2. Remove this
warning prior to submitting.}

\subsection{Deconstruct}\label{deconstruct}

\subsubsection{Original}\label{original}

The original data visualisation selected for the assignment was as
follows:

\emph{Source: Newsweek, ``Most Religious Countries in the World.}

\subsubsection{Objective and Audience}\label{objective-and-audience}

The objective and audience of the original data visualisation chosen can
be summarised as follows:

\textbf{Objective}

News websites commonly use maps as an engagement tool. There might have
been not much thought made about the best kind of data visualisation to
use or what the objective of this map was. The author of this report has
reason to believe this is true, based on years of experience working in
newsrooms.

Assuming that the authors had an objective, then the objective was to
represent the geographic distribution of religiosity worldwide.

\textbf{Audience}

Newsweek is a general news website, so the audience is a general and
non-technical audience. Newsweek has news from around the world and is
written for English-speaking audiences. We can expect the majority of
readers will live in developed western countries.

\subsubsection{Critique}\label{critique}

The visualisation chosen had the following three main issues:

\begin{itemize}
\tightlist
\item
  The visualisation uses a palette of colours from across the colour
  spectrum, which would be appropriate for categorical data. However,
  the variable of interest is continuous and placed in levels, meaning
  that these levels have order. A sequential colour scale would be more
  appropriate.
\item
  The mapping from levels in the data to colours is confusing: ``No
  Data'' has a colour and is in the palette's middle. The palette also
  does not observe any sensible order in relation to the levels of the
  variable of interest, which is evident from looking at the legend.
\item
  A map is not the right choice for this data. Many large landmasses
  have no data, meaning there are large gaps in the data visualisation.
  Also, it is somewhat interesting to see that some areas of Earth are
  more religious than others, but it would be more interesting to see if
  religiosity correlates with some other variable other than location on
  Earth.
\end{itemize}

\subsection{Reconstruct}\label{reconstruct}

This visualisation would be more interesting if religiosity was plotted
against some other variable, like some measure of poverty.

Using a scatter plot to see a correlation would address the issue of
``No Data'' in the original map, since scatter plots do not alert the
audience to missing data. But a note on the visualisation should explain
that some countries do not have available data. To draw the reader's
attention to the correlation, each chart can have a ``line of best
fit''. Most readers in a general audience will know what this means.

One issue with the visualisation that was not previously mentioned is it
does not link to the data source, but only names Pew Research Centre.
The most likely source is a report by Pew in June of 2024. The data in
that report is displayed in a horizontal bar chart with vertical
gridlines at 20\%, 40\%, 60\% and 80\%, which match the levels of the
Newsweek map's data.

That Pew Research data was unavailable so instead this reconstruction
uses another Pew Research source, which was published in 2022 and
includes data from 2010. That data includes ``unaffiliated'' to mean
those people who do not follow a religion. The remainder
(100-unaffiliated)\% should be close to the notion of ``importance of
religion'' that the Newsweek map displays.

Of particular interest would be this question: does religiosity
correlate with poverty? To achieve this, we will combine that data with
another dataset. A faceted data visualisation will allow for
consideration of many levels of poverty severity.

\subsubsection{Code}\label{code}

The following code was used to fix the issues identified in the
original. Note that a small amount of data cleaning and prep was done in
MS Excel.

\begin{Shaded}
\begin{Highlighting}[]
\NormalTok{poverty }\OtherTok{=} \FunctionTok{read.csv}\NormalTok{(}\StringTok{"poverty{-}explorer.csv"}\NormalTok{)}

\NormalTok{poverty[, }\StringTok{"Share.10.to.20.a.day"}\NormalTok{] }\OtherTok{\textless{}{-}}\NormalTok{ poverty[, }\StringTok{"Share.below..20.a.day"}\NormalTok{] }\SpecialCharTok{{-}}
\NormalTok{  poverty[, }\StringTok{"Share.below..10.a.day"}\NormalTok{]}
\NormalTok{poverty[, }\StringTok{"Share.20.to.30.a.day"}\NormalTok{] }\OtherTok{\textless{}{-}}\NormalTok{ poverty[, }\StringTok{"Share.below..30.a.day"}\NormalTok{] }\SpecialCharTok{{-}}
\NormalTok{  poverty[, }\StringTok{"Share.below..20.a.day"}\NormalTok{]}
\NormalTok{poverty[, }\StringTok{"Share.30.to.40.a.day"}\NormalTok{] }\OtherTok{\textless{}{-}}\NormalTok{ poverty[, }\StringTok{"Share.below..40.a.day"}\NormalTok{] }\SpecialCharTok{{-}}
\NormalTok{  poverty[, }\StringTok{"Share.below..30.a.day"}\NormalTok{]}

\NormalTok{poverty2008To2012 }\OtherTok{=}\NormalTok{ poverty[poverty}\SpecialCharTok{$}\NormalTok{Year }\SpecialCharTok{\%in\%} \FunctionTok{c}\NormalTok{(}\StringTok{\textquotesingle{}2008\textquotesingle{}}\NormalTok{,}\StringTok{\textquotesingle{}2009\textquotesingle{}}\NormalTok{,}\StringTok{\textquotesingle{}2010\textquotesingle{}}\NormalTok{,}\StringTok{\textquotesingle{}2011\textquotesingle{}}\NormalTok{,}\StringTok{\textquotesingle{}2012\textquotesingle{}}\NormalTok{), }\FunctionTok{c}\NormalTok{(}\StringTok{"Country"}\NormalTok{, }\StringTok{"Year"}\NormalTok{, }\StringTok{"Share.below..10.a.day"}\NormalTok{, }\StringTok{"Share.10.to.20.a.day"}\NormalTok{, }\StringTok{"Share.20.to.30.a.day"}\NormalTok{, }\StringTok{"Share.30.to.40.a.day"}\NormalTok{)]}

\CommentTok{\# Collect data from 2010, but take the closes year if 2010 is not available.}
\NormalTok{order }\OtherTok{\textless{}{-}} \FunctionTok{c}\NormalTok{(}\StringTok{\textquotesingle{}2010\textquotesingle{}}\NormalTok{, }\StringTok{\textquotesingle{}2011\textquotesingle{}}\NormalTok{, }\StringTok{\textquotesingle{}2009\textquotesingle{}}\NormalTok{, }\StringTok{\textquotesingle{}2012\textquotesingle{}}\NormalTok{, }\StringTok{\textquotesingle{}2008\textquotesingle{}}\NormalTok{)}
\FunctionTok{library}\NormalTok{(dplyr)}
\NormalTok{povertyClosestYear }\OtherTok{\textless{}{-}}\NormalTok{ poverty2008To2012 }\SpecialCharTok{\%\textgreater{}\%} 
  \FunctionTok{arrange}\NormalTok{(}\FunctionTok{factor}\NormalTok{(Year, }\AttributeTok{levels =}\NormalTok{ order)) }\SpecialCharTok{\%\textgreater{}\%}
  \FunctionTok{distinct}\NormalTok{(Country, }\AttributeTok{.keep\_all =} \ConstantTok{TRUE}\NormalTok{)}

\NormalTok{religion2010 }\OtherTok{=} \FunctionTok{read.csv}\NormalTok{(}\StringTok{"ReligionPerCountry\_2010.csv"}\NormalTok{)}
\NormalTok{religion2010}\SpecialCharTok{$}\NormalTok{Unaffiliated }\OtherTok{=} \FunctionTok{gsub}\NormalTok{(}\StringTok{\textquotesingle{}[\textless{}\%]\textquotesingle{}}\NormalTok{, }\StringTok{\textquotesingle{}\textquotesingle{}}\NormalTok{, religion2010}\SpecialCharTok{$}\NormalTok{Unaffiliated)}
\NormalTok{religion2010}\SpecialCharTok{$}\NormalTok{Unaffiliated }\OtherTok{=} \FunctionTok{as.numeric}\NormalTok{(religion2010}\SpecialCharTok{$}\NormalTok{Unaffiliated)}
\NormalTok{religion2010}\SpecialCharTok{$}\NormalTok{Religious }\OtherTok{=} \DecValTok{100} \SpecialCharTok{{-}}\NormalTok{ religion2010}\SpecialCharTok{$}\NormalTok{Unaffiliated}
\NormalTok{religion2010 }\OtherTok{=}\NormalTok{ religion2010[,}\FunctionTok{c}\NormalTok{(}\StringTok{"Country"}\NormalTok{,}\StringTok{"Religious"}\NormalTok{)]}

\NormalTok{mergedData }\OtherTok{=} \FunctionTok{merge}\NormalTok{(povertyClosestYear, religion2010, }\AttributeTok{by =} \StringTok{"Country"}\NormalTok{)}
\NormalTok{mergedData }\OtherTok{\textless{}{-}}\NormalTok{ mergedData[,}\FunctionTok{c}\NormalTok{(}\StringTok{"Share.below..10.a.day"}\NormalTok{, }
                            \StringTok{"Share.10.to.20.a.day"}\NormalTok{, }
                            \StringTok{"Share.20.to.30.a.day"}\NormalTok{, }
                            \StringTok{"Share.30.to.40.a.day"}\NormalTok{,}
                            \StringTok{"Religious"}\NormalTok{)]}

\CommentTok{\# Make data long for ggplot}
\FunctionTok{library}\NormalTok{(tidyr)}
\NormalTok{mergedDataLong }\OtherTok{\textless{}{-}}\NormalTok{ mergedData }\SpecialCharTok{\%\textgreater{}\%} 
  \FunctionTok{gather}\NormalTok{(IncomeBracket, ShareInBracket, }\SpecialCharTok{{-}}\NormalTok{Religious)}

\NormalTok{variable\_names }\OtherTok{\textless{}{-}} \FunctionTok{list}\NormalTok{(}
  \StringTok{"Share.below..10.a.day"} \OtherTok{=} \StringTok{"$0 to $10 per day"}\NormalTok{ ,}
  \StringTok{"Share.10.to.20.a.day"} \OtherTok{=} \StringTok{"$10 to $20 per day"}\NormalTok{,}
  \StringTok{"Share.20.to.30.a.day"} \OtherTok{=} \StringTok{"$20 to $30 per day"}\NormalTok{,}
  \StringTok{"Share.30.to.40.a.day"} \OtherTok{=} \StringTok{"$30 to $40 per day"}
\NormalTok{)}
\NormalTok{variable\_labeller }\OtherTok{\textless{}{-}} \ControlFlowTok{function}\NormalTok{(variable,value)\{}
  \FunctionTok{return}\NormalTok{(variable\_names[value])}
\NormalTok{\}}
\end{Highlighting}
\end{Shaded}

\subsubsection{Reconstruction}\label{reconstruction}

The following plot fixes the main issues in the original.

The chart shows that there is a strong correlation between religiosity
and poverty, but the correlation between religiosity and poverty
reverses somewhat when we consider less severe poverty.

The code here is non-trivial, and is echoed in case it is of interest.

\begin{Shaded}
\begin{Highlighting}[]
\FunctionTok{library}\NormalTok{(}\StringTok{"ggplot2"}\NormalTok{)}

\FunctionTok{ggplot}\NormalTok{(mergedDataLong, }\FunctionTok{aes}\NormalTok{(}\AttributeTok{x=}\NormalTok{Religious, }\AttributeTok{y=}\NormalTok{ShareInBracket)) }\SpecialCharTok{+}
  \FunctionTok{geom\_point}\NormalTok{(}\AttributeTok{size =} \FloatTok{0.5}\NormalTok{) }\SpecialCharTok{+}
  \FunctionTok{theme}\NormalTok{(}\AttributeTok{aspect.ratio =} \DecValTok{1}\NormalTok{,}
        \AttributeTok{plot.title =} \FunctionTok{element\_text}\NormalTok{(}\AttributeTok{hjust =} \FloatTok{0.5}\NormalTok{,}\AttributeTok{size =} \DecValTok{15}\NormalTok{),}
        \AttributeTok{plot.caption =} \FunctionTok{element\_text}\NormalTok{(}\AttributeTok{hjust =} \DecValTok{0}\NormalTok{)) }\SpecialCharTok{+}
  \FunctionTok{labs}\NormalTok{(}\AttributeTok{x=} \StringTok{"Religious (\%)"}\NormalTok{, }
       \AttributeTok{y=}\StringTok{"People who live in income bracket (\%)"}\NormalTok{,}
       \AttributeTok{title=} \StringTok{"Countries by poverty and religiosity, 2010"}\NormalTok{,}
       \AttributeTok{caption =} \StringTok{"Notes:}\SpecialCharTok{\textbackslash{}n}\StringTok{{-} Data for some countries was missing.}\SpecialCharTok{\textbackslash{}n}\StringTok{{-} Income per day was adjusted for inflation and cost of living.}\SpecialCharTok{\textbackslash{}n}\StringTok{{-} In some cases the closest possible year to 2010 was used.}\SpecialCharTok{\textbackslash{}n}\StringTok{Data:}\SpecialCharTok{\textbackslash{}n}\StringTok{{-} \textquotesingle{}Poverty Data Explorer\textquotesingle{}, Our World in Data}\SpecialCharTok{\textbackslash{}n}\StringTok{{-} \textquotesingle{}The Global Religious Landscape\textquotesingle{}, Pew Research Center"}\NormalTok{) }\SpecialCharTok{+}
  \FunctionTok{ylim}\NormalTok{(}\DecValTok{0}\NormalTok{,}\DecValTok{100}\NormalTok{) }\SpecialCharTok{+}
  \FunctionTok{xlim}\NormalTok{(}\DecValTok{0}\NormalTok{,}\DecValTok{100}\NormalTok{) }\SpecialCharTok{+}
  \FunctionTok{geom\_smooth}\NormalTok{(}\AttributeTok{method=}\NormalTok{lm, }\AttributeTok{se=}\ConstantTok{FALSE}\NormalTok{, }\AttributeTok{linetype=}\StringTok{"solid"}\NormalTok{, }\AttributeTok{color=}\StringTok{"darkred"}\NormalTok{, }\AttributeTok{linewidth =} \FloatTok{0.5}\NormalTok{) }\SpecialCharTok{+}
  \FunctionTok{facet\_wrap}\NormalTok{(}\SpecialCharTok{\textasciitilde{}}\FunctionTok{factor}\NormalTok{(IncomeBracket, }\AttributeTok{levels=}\FunctionTok{c}\NormalTok{(}\StringTok{"Share.below..10.a.day"}\NormalTok{,}
                                             \StringTok{"Share.10.to.20.a.day"}\NormalTok{,}
                                             \StringTok{"Share.20.to.30.a.day"}\NormalTok{,}
                                             \StringTok{"Share.30.to.40.a.day"}\NormalTok{)),}
             \AttributeTok{scales=}\StringTok{"free\_y"}\NormalTok{, }\AttributeTok{ncol=}\DecValTok{2}\NormalTok{, }\AttributeTok{labeller=}\NormalTok{ variable\_labeller)}
\end{Highlighting}
\end{Shaded}

\begin{center}\includegraphics{Assignment_2_RMarkdown_Template_files/figure-latex/unnamed-chunk-2-1} \end{center}

\subsection{References}\label{references}

The reference to the original data visualisation, the data sources used
for the reconstruction are as follows:

\begin{itemize}
\tightlist
\item
  \url{https://www.newsweek.com/map-shows-most-religious-countries-2024-1942346}
\item
  likely data source
  \url{https://www.pewresearch.org/religion/2024/06/17/religion-and-spirituality-in-east-asian-societies/}
\item
  new data source
  \url{https://assets.pewresearch.org/wp-content/uploads/sites/11/2014/01/global-religion-full.pdf}
\item
  poverty data
  \url{https://ourworldindata.org/explorers/poverty-explorer}
\end{itemize}

\end{document}
